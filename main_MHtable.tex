% Copyright 2004 by Till Tantau <tantau@users.sourceforge.net>.
%
% In principle, this file can be redistributed and/or modified under
% the terms of the GNU Public License, version 2.
%
% However, this file is supposed to be a template to be modified
% for your own needs. For this reason, if you use this file as a
% template and not specifically distribute it as part of a another
% package/program, I grant the extra permission to freely copy and
% modify this file as you see fit and even to delete this copyright
% notice. 

\documentclass{beamer}

% There are many different themes available for Beamer. A comprehensive
% list with examples is given here:
% http://deic.uab.es/~iblanes/beamer_gallery/index_by_theme.html
% You can uncomment the themes below if you would like to use a different
% one:
%\usetheme{AnnArbor}
%\usetheme{Antibes}
%\usetheme{Bergen}
%\usetheme{Berkeley}
%\usetheme{Berlin}
%\usetheme{Boadilla}
%\usetheme{boxes}
%\usetheme{CambridgeUS}
%\usetheme{Copenhagen}
%\usetheme{Darmstadt}
\usetheme{default}
%\usetheme{Frankfurt}
%\usetheme{Goettingen}
%\usetheme{Hannover}
%\usetheme{Ilmenau}
%\usetheme{JuanLesPins}
%\usetheme{Luebeck}
%\usetheme{Madrid}
%\usetheme{Malmoe}
%\usetheme{Marburg}
%\usetheme{Montpellier}
%\usetheme{PaloAlto}
%\usetheme{Pittsburgh}
%\usetheme{Rochester}
%\usetheme{Singapore}
%\usetheme{Szeged}
%\usetheme{Warsaw}

\title{Presentation Title}

% A subtitle is optional and this may be deleted
\subtitle{Optional Subtitle}

\author{F.~Author\inst{1} \and S.~Another\inst{2}}
% - Give the names in the same order as the appear in the paper.
% - Use the \inst{?} command only if the authors have different
%   affiliation.

\institute[Universities of Somewhere and Elsewhere] % (optional, but mostly needed)
{
  \inst{1}%
  Department of Computer Science\\
  University of Somewhere
  \and
  \inst{2}%
  Department of Theoretical Philosophy\\
  University of Elsewhere}
% - Use the \inst command only if there are several affiliations.
% - Keep it simple, no one is interested in your street address.

\date{Conference Name, 2013}
% - Either use conference name or its abbreviation.
% - Not really informative to the audience, more for people (including
%   yourself) who are reading the slides online

\subject{Theoretical Computer Science}
% This is only inserted into the PDF information catalog. Can be left
% out. 

% If you have a file called "university-logo-filename.xxx", where xxx
% is a graphic format that can be processed by latex or pdflatex,
% resp., then you can add a logo as follows:

% \pgfdeclareimage[height=0.5cm]{university-logo}{university-logo-filename}
% \logo{\pgfuseimage{university-logo}}

% Delete this, if you do not want the table of contents to pop up at
% the beginning of each subsection:
\AtBeginSubsection[]
{
  \begin{frame}<beamer>{Outline}
    \tableofcontents[currentsection,currentsubsection]
  \end{frame}
}

% Let's get started
\begin{document}

\begin{frame}
  \titlepage
\end{frame}

\begin{frame}{Outline}
  \tableofcontents
  % You might wish to add the option [pausesections]
\end{frame}

% Section and subsections will appear in the presentation overview
% and table of contents.
\section{First Main Section}

\subsection{First Subsection}

\begin{frame}{First Slide Title}{Optional Subtitle}
\begin{table}[htbp]
  \centering\scriptsize
  \begin{tabular}{*{3}{l}*{5}{r}}
    \toprule
    deltaTreat & beta12 & cs \textbar\ \( \rho \) & \multicolumn{1}{c}{-0.5} & \multicolumn{1}{c}{-0.25} & \multicolumn{1}{c}{0} & \multicolumn{1}{c}{0.25} & \multicolumn{1}{c}{0.5} \\
    \midrule
    0 & 0 & 1 & -0.02 & -0.01 & -0.02 & -0.05 & 0.02 \\
    &  & 0.8 & -0.01 & -0.03 & -0.08 & -0.01 & 0.02 \\
    &  & 0.6 & -0.01 & -0.02 & -0.01 & -0.06 & 0.00 \\ \addlinespace[3pt]
    & 1 & 1 & 1.00 & 0.98 & 0.89 & 0.88 & 1.00 \\
    &  & 0.8 & 0.97 & 0.92 & 0.94 & 0.98 & 1.06 \\
    &  & 0.6 & 0.95 & 0.93 & 0.87 & 0.81 & 1.06 \\ \addlinespace[6pt]
    0.5 & 0 & 1 & -0.01 & -0.04 & -0.02 & -0.03 & 0.01 \\
    &  & 0.8 & 0.00 & -0.01 & -0.04 & -0.02 & 0.01 \\
    &  & 0.6 & -0.03 & 0.01 & -0.05 & 0.01 & -0.00 \\ \addlinespace[3pt]
    & 1 & 1 & 1.01 & 0.99 & 1.02 & 0.91 & 1.03 \\
    &  & 0.8 & 0.99 & 0.96 & 0.95 & 0.90 & 1.01 \\
    &  & 0.6 & 1.01 & 0.96 & 0.95 & 0.98 & 1.00 \\
    \bottomrule
  \end{tabular}
  \caption{Table 3, n=1000}
  \label{tab:ft}
\end{table}

\end{frame}

\begin{frame}{Table 3 (ctd)}

 \begin{table}[htbp]
  \centering\scriptsize
  \begin{tabular}{*{3}{l}*{5}{r}}
    \toprule
    deltaTreat & beta12 & cs \textbar\ \( \rho \) & \multicolumn{1}{c}{-0.5} & \multicolumn{1}{c}{-0.25} & \multicolumn{1}{c}{0} & \multicolumn{1}{c}{0.25} & \multicolumn{1}{c}{0.5} \\
    \midrule
    0 & 0 & 1 & -0.02 & -0.01 & -0.02 & -0.05 & 0.02 \\
    &  & 0.8 & -0.01 & -0.03 & -0.08 & -0.01 & 0.02 \\
    &  & 0.6 & -0.01 & -0.02 & -0.01 & -0.06 & 0.00 \\ \addlinespace[3pt]
    & 1 & 1 & 1.00 & 0.98 & 0.89 & 0.88 & 1.00 \\
    &  & 0.8 & 0.97 & 0.92 & 0.94 & 0.98 & 1.06 \\
    &  & 0.6 & 0.95 & 0.93 & 0.87 & 0.81 & 1.06 \\ \addlinespace[6pt]
    0.5 & 0 & 1 & -0.01 & -0.04 & -0.02 & -0.03 & 0.01 \\
    &  & 0.8 & 0.00 & -0.01 & -0.04 & -0.02 & 0.01 \\
    &  & 0.6 & -0.03 & 0.01 & -0.05 & 0.01 & -0.00 \\ \addlinespace[3pt]
    & 1 & 1 & 1.01 & 0.99 & 1.02 & 0.91 & 1.03 \\
    &  & 0.8 & 0.99 & 0.96 & 0.95 & 0.90 & 1.01 \\
    &  & 0.6 & 1.01 & 0.96 & 0.95 & 0.98 & 1.00 \\
    \bottomrule
  \end{tabular}
  \caption{Table 3, n=1000, alternative format}
  \label{tab:ft2}
\end{table} 

\end{frame}


\subsection{Second Subsection}

% You can reveal the parts of a slide one at a time
% with the \pause command:
\begin{frame}{Second Slide Title}
  \begin{itemize}
  \item {
    First item.
    \pause % The slide will pause after showing the first item
  }
  \item {   
    Second item.
  }
  % You can also specify when the content should appear
  % by using <n->:
  \item<3-> {
    Third item.
  }
  \item<4-> {
    Fourth item.
  }
  % or you can use the \uncover command to reveal general
  % content (not just \items):
  \item<5-> {
    Fifth item. \uncover<6->{Extra text in the fifth item.}
  }
  \end{itemize}
\end{frame}

\section{Second Main Section}

\subsection{Another Subsection}

\begin{frame}{Blocks}
\begin{block}{Block Title}
You can also highlight sections of your presentation in a block, with it's own title
\end{block}
\begin{theorem}
There are separate environments for theorems, examples, definitions and proofs.
\end{theorem}
\begin{example}
Here is an example of an example block.
\end{example}
\end{frame}

% Placing a * after \section means it will not show in the
% outline or table of contents.
\section*{Summary}

\begin{frame}{Summary}
  \begin{itemize}
  \item
    The \alert{first main message} of your talk in one or two lines.
  \item
    The \alert{second main message} of your talk in one or two lines.
  \item
    Perhaps a \alert{third message}, but not more than that.
  \end{itemize}
  
  \begin{itemize}
  \item
    Outlook
    \begin{itemize}
    \item
      Something you haven't solved.
    \item
      Something else you haven't solved.
    \end{itemize}
  \end{itemize}
\end{frame}



% All of the following is optional and typically not needed. 
\appendix
\section<presentation>*{\appendixname}
\subsection<presentation>*{For Further Reading}

\begin{frame}[allowframebreaks]
  \frametitle<presentation>{For Further Reading}
    
  \begin{thebibliography}{10}
    
  \beamertemplatebookbibitems
  % Start with overview books.

  \bibitem{Author1990}
    A.~Author.
    \newblock {\em Handbook of Everything}.
    \newblock Some Press, 1990.
 
    
  \beamertemplatearticlebibitems
  % Followed by interesting articles. Keep the list short. 

  \bibitem{Someone2000}
    S.~Someone.
    \newblock On this and that.
    \newblock {\em Journal of This and That}, 2(1):50--100,
    2000.
  \end{thebibliography}
\end{frame}

\end{document}


