\documentclass{beamer}
\usepackage{booktabs}
% There are many different themes available for Beamer. A comprehensive
% list with examples is given here:
% http://deic.uab.es/~iblanes/beamer_gallery/index_by_theme.html
% You can uncomment the themes below if you would like to use a different
% one:
%\usetheme{AnnArbor}
%\usetheme{Antibes}
%\usetheme{Bergen}
%\usetheme{Berkeley}
%\usetheme{Berlin}
%\usetheme{Boadilla}
%\usetheme{boxes}
%\usetheme{CambridgeUS}
%\usetheme{Copenhagen}
%\usetheme{Darmstadt}
\usetheme{default}
%\usetheme{Frankfurt}
%\usetheme{Goettingen}
%\usetheme{Hannover}
%\usetheme{Ilmenau}
%\usetheme{JuanLesPins}
%\usetheme{Luebeck}
%\usetheme{Madrid}
%\usetheme{Malmoe}
%\usetheme{Marburg}
%\usetheme{Montpellier}
%\usetheme{PaloAlto}
%\usetheme{Pittsburgh}
%\usetheme{Rochester}
%\usetheme{Singapore}
%\usetheme{Szeged}
%\usetheme{Warsaw}

\title{Correlated bivariate Normal competing risks%  -- simulation findings in an ill-posed problem
}

% A subtitle is optional and this may be deleted
\subtitle{ -- simulation findings in an ill-posed problem}

\author{Malcolm~Hudson \and V.~Gares \and M.~Manuguerra \and V.~Gebski}
% - Give the names in the same order as the appear in the paper.
% - Use the \inst{?} command only if the authors have different
%   affiliation.

\institute[Macquarie University] % (optional, but mostly needed)

\date{ISCB ASC 2018, Melbourne}

\subject{Competing Risks for bivariate Normal}
% This is only inserted into the PDF information catalog. Can be left
% out. 

% If you have a file called "university-logo-filename.xxx", where xxx
% is a graphic format that can be processed by latex or pdflatex,
% resp., then you can add a logo as follows:

\pgfdeclareimage[height=0.5cm]{MU-logo}{MQlogo.png}
\logo{\pgfuseimage{MU-logo}}

% Let's get started
\begin{document}

\begin{frame}
  \titlepage
\end{frame}

\begin{frame}{Outline}
  \tableofcontents
  % You might wish to add the option [pausesections]
\end{frame}

% Section and subsections will appear in the presentation overview
% and table of contents.
\section{First Main Section}

\subsection{First Subsection}

\begin{frame}{Table 1a (n=1000)}{Estimation: $\hat{\rho}$}

\begin{table}[htbp]
  \centering\scriptsize
  \begin{tabular}{*{2}{l}*{3}{r}}
    \toprule
    cs & \( \rho \) \textbar\ beta2 & \multicolumn{1}{c}{0} & \multicolumn{1}{c}{0.5} & \multicolumn{1}{c}{1} \\
    \midrule
    1 & -0.5 & -0.48 & -0.49 & -0.49 \\
    & -0.25 & -0.22 & -0.23 & -0.21 \\
    & 0 & 0.04 & 0.04 & 0.05 \\
    & 0.25 & 0.32 & 0.32 & 0.34 \\
    & 0.5 & 0.49 & 0.47 & 0.48 \\ \addlinespace[3pt]
    0.8 & -0.5 & -0.48 & -0.48 & -0.47 \\
    & -0.25 & -0.22 & -0.21 & -0.19 \\
    & 0 & 0.08 & 0.09 & 0.13 \\
    & 0.25 & 0.32 & 0.37 & 0.35 \\
    & 0.5 & 0.48 & 0.48 & 0.48 \\ \addlinespace[3pt]
    0.6 & -0.5 & -0.48 & -0.47 & -0.46 \\
    & -0.25 & -0.24 & -0.20 & -0.09 \\
    & 0 & 0.10 & 0.13 & 0.22 \\
    & 0.25 & 0.35 & 0.38 & 0.39 \\
    & 0.5 & 0.48 & 0.48 & 0.47 \\
    \bottomrule
  \end{tabular}
  \caption{median of 100 replicates}
  \label{tab:1ft}
\end{table}

\end{frame}

\begin{frame}{Table 1b (n=1000)}{squareEM iterations}

\begin{table}[htbp]
  \centering\scriptsize
  \begin{tabular}{*{2}{l}*{3}{r}}
    \toprule
    cs & \( \rho \) \textbar\ beta2 & \multicolumn{1}{c}{0} & \multicolumn{1}{c}{0.5} & \multicolumn{1}{c}{1} \\
    \midrule
    1 & -0.5 & 36 & 95 & 93 \\
    & -0.25 & 83 & 107 & 154 \\
    & 0 & 107 & 142 & 250 \\
    & 0.25 & 90 & 127 & 250 \\
    & 0.5 & 112 & 227 & 250 \\ \addlinespace[3pt]
    0.8 & -0.5 & 36 & 90 & 116 \\
    & -0.25 & 88 & 108 & 243 \\
    & 0 & 119 & 163 & 250 \\
    & 0.25 & 146 & 219 & 250 \\
    & 0.5 & 141 & 248 & 250 \\ \addlinespace[3pt]
    0.6 & -0.5 & 78 & 105 & 250 \\
    & -0.25 & 117 & 250 & 235 \\
    & 0 & 134 & 212 & 250 \\
    & 0.25 & 159 & 240 & 250 \\
    & 0.5 & 184 & 250 & 250 \\
    \bottomrule
  \end{tabular}
  \caption{max number iterations to converge}
  \label{tab:ft1b}
\end{table}
\end{frame}



\begin{frame}{Table 3a (n=1000)}{Estimation: $\hat{\beta}_{12}$}
\begin{table}[htbp]
  \centering\scriptsize
  \begin{tabular}{*{2}{l}*{4}{r}}
    \toprule
     & beta12 & \multicolumn{2}{c}{0} & \multicolumn{2}{c}{1} \\
    \cmidrule(lr){3-4} \cmidrule(lr){5-6}
    cs & \( \rho \) \textbar\ deltaTreat & \multicolumn{1}{c}{0} & \multicolumn{1}{c}{0.5} & \multicolumn{1}{c}{0} & \multicolumn{1}{c}{0.5} \\
    \midrule
    1 & -0.5 & -0.02 & -0.01 & 1.00 & 1.01 \\
    & -0.25 & -0.01 & -0.04 & 0.98 & 0.99 \\
    & 0 & -0.02 & -0.02 & 0.89 & 1.02 \\
    & 0.25 & -0.05 & -0.03 & 0.88 & 0.91 \\
    & 0.5 & 0.02 & 0.01 & 1.00 & 1.03 \\ \addlinespace[3pt]
    0.8 & -0.5 & -0.01 & 0.00 & 0.97 & 0.99 \\
    & -0.25 & -0.03 & -0.01 & 0.92 & 0.96 \\
    & 0 & -0.08 & -0.04 & 0.94 & 0.95 \\
    & 0.25 & -0.01 & -0.02 & 0.98 & 0.90 \\
    & 0.5 & 0.02 & 0.01 & 1.06 & 1.01 \\ \addlinespace[3pt]
    0.6 & -0.5 & -0.01 & -0.03 & 0.95 & 1.01 \\
    & -0.25 & -0.02 & 0.01 & 0.93 & 0.96 \\
    & 0 & -0.01 & -0.05 & 0.87 & 0.95 \\
    & 0.25 & -0.06 & 0.01 & 0.81 & 0.98 \\
    & 0.5 & 0.00 & -0.00 & 1.06 & 1.00 \\
    \bottomrule
  \end{tabular}
%  \caption{Table 3, n=1000}
  \label{tab:ft}
\end{table}

\end{frame}


\begin{frame}{Table 3b (n=1000)}{Estimates: Treatment $\hat{\Delta}=\hat{\beta}_{21}$}
\begin{table}[htbp]
  \centering\scriptsize
  \begin{tabular}{*{2}{l}*{4}{r}}
    \toprule
     & beta12 & \multicolumn{2}{c}{0} & \multicolumn{2}{c}{1} \\
    \cmidrule(lr){3-4} \cmidrule(lr){5-6}
    cs & \( \rho \) \textbar\ deltaTreat & \multicolumn{1}{c}{0} & \multicolumn{1}{c}{0.5} & \multicolumn{1}{c}{0} & \multicolumn{1}{c}{0.5} \\
    \midrule
    1 & -0.5 & -0.00 & 0.50 & 0.00 & 0.51 \\
    & -0.25 & -0.00 & 0.48 & -0.01 & 0.50 \\
    & 0 & 0.00 & 0.46 & 0.00 & 0.50 \\
    & 0.25 & -0.01 & 0.49 & 0.00 & 0.49 \\
    & 0.5 & -0.00 & 0.52 & -0.00 & 0.49 \\ \addlinespace[3pt]
    0.8 & -0.5 & -0.00 & 0.50 & 0.00 & 0.50 \\
    & -0.25 & -0.00 & 0.48 & 0.01 & 0.49 \\
    & 0 & 0.01 & 0.49 & 0.00 & 0.49 \\
    & 0.25 & -0.00 & 0.48 & -0.01 & 0.48 \\
    & 0.5 & -0.01 & 0.51 & 0.01 & 0.51 \\ \addlinespace[3pt]
    0.6 & -0.5 & 0.02 & 0.49 & -0.01 & 0.48 \\
    & -0.25 & -0.02 & 0.52 & -0.00 & 0.48 \\
    & 0 & -0.01 & 0.49 & 0.01 & 0.48 \\
    & 0.25 & 0.02 & 0.49 & 0.01 & 0.50 \\
    & 0.5 & 0.00 & 0.51 & -0.01 & 0.51 \\
    \bottomrule
  \end{tabular}
%  \caption{b21 TreatDiff estimate: Table 3b, n=1000}
  \label{tab:ft21}
\end{table}

\end{frame}

\begin{frame}{Table 3c (n=1000)}{Estimates: Treatment $\hat{\rho}$}
\begin{table}[htbp]
  \centering\scriptsize
  \begin{tabular}{*{2}{l}*{4}{r}}
    \toprule
     & beta12 & \multicolumn{2}{c}{0} & \multicolumn{2}{c}{1} \\
    \cmidrule(lr){3-4} \cmidrule(lr){5-6}
    cs & \( \rho \) \textbar\ deltaTreat & \multicolumn{1}{c}{0} & \multicolumn{1}{c}{0.5} & \multicolumn{1}{c}{0} & \multicolumn{1}{c}{0.5} \\
    \midrule
    1 & -0.5 & -0.50 & -0.49 & -0.51 & -0.51 \\
    & -0.25 & -0.24 & -0.22 & -0.23 & -0.25 \\
    & 0 & 0.05 & 0.07 & 0.12 & -0.02 \\
    & 0.25 & 0.36 & 0.31 & 0.37 & 0.31 \\
    & 0.5 & 0.49 & 0.49 & 0.51 & 0.50 \\ \addlinespace[3pt]
    0.8 & -0.5 & -0.50 & -0.49 & -0.48 & -0.50 \\
    & -0.25 & -0.20 & -0.22 & -0.16 & -0.24 \\
    & 0 & 0.17 & 0.07 & 0.10 & 0.02 \\
    & 0.25 & 0.31 & 0.27 & 0.25 & 0.37 \\
    & 0.5 & 0.51 & 0.48 & 0.48 & 0.49 \\ \addlinespace[3pt]
    0.6 & -0.5 & -0.48 & -0.48 & -0.47 & -0.49 \\
    & -0.25 & -0.19 & -0.25 & -0.22 & -0.19 \\
    & 0 & 0.03 & 0.09 & 0.14 & 0.05 \\
    & 0.25 & 0.40 & 0.27 & 0.44 & 0.27 \\
    & 0.5 & 0.53 & 0.50 & 0.46 & 0.51 \\
    \bottomrule
  \end{tabular}
  \caption{Correlation estimate (median): n=1000, nsim=100}
  \label{tab:ft3}
\end{table}

\end{frame}


\section{Second Main Section}

\subsection{Another Subsection}

\begin{frame}{Blocks}
\begin{block}{Block Title}
You can also highlight sections of your presentation in a block, with it's own title
\end{block}
\begin{theorem}
There are separate environments for theorems, examples, definitions and proofs.
\end{theorem}
\begin{example}
Here is an example of an example block.
\end{example}
\end{frame}

% Placing a * after \section means it will not show in the
% outline or table of contents.
\section*{Summary}

\begin{frame}{Summary}
  \begin{itemize}
  \item
    The \alert{first main message} of your talk in one or two lines.
  \item
    The \alert{second main message} of your talk in one or two lines.
  \item
    Perhaps a \alert{third message}, but not more than that.
  \end{itemize}
  
  \begin{itemize}
  \item
    Outlook
    \begin{itemize}
    \item
      Something you haven't solved.
    \item
      Something else you haven't solved.
    \end{itemize}
  \end{itemize}
\end{frame}



% All of the following is optional and typically not needed. 
\appendix
\section<presentation>*{\appendixname}
\subsection<presentation>*{For Further Reading}

\begin{frame}[allowframebreaks]
  \frametitle<presentation>{For Further Reading}
    
  \begin{thebibliography}{10}
    
  \beamertemplatebookbibitems
  % Start with overview books.

  \bibitem{Author1990}
    A.~Author.
    \newblock {\em Handbook of Everything}.
    \newblock Some Press, 1990.
 
    
  \beamertemplatearticlebibitems
  % Followed by interesting articles. Keep the list short. 

  \bibitem{Someone2000}
    S.~Someone.
    \newblock On this and that.
    \newblock {\em Journal of This and That}, 2(1):50--100,
    2000.
  \end{thebibliography}
\end{frame}

\end{document}